% Options for packages loaded elsewhere
\PassOptionsToPackage{unicode}{hyperref}
\PassOptionsToPackage{hyphens}{url}
%
\documentclass[
]{article}
\usepackage{amsmath,amssymb}
\usepackage{iftex}
\ifPDFTeX
  \usepackage[T1]{fontenc}
  \usepackage[utf8]{inputenc}
  \usepackage{textcomp} % provide euro and other symbols
\else % if luatex or xetex
  \usepackage{unicode-math} % this also loads fontspec
  \defaultfontfeatures{Scale=MatchLowercase}
  \defaultfontfeatures[\rmfamily]{Ligatures=TeX,Scale=1}
\fi
\usepackage{lmodern}
\ifPDFTeX\else
  % xetex/luatex font selection
\fi
% Use upquote if available, for straight quotes in verbatim environments
\IfFileExists{upquote.sty}{\usepackage{upquote}}{}
\IfFileExists{microtype.sty}{% use microtype if available
  \usepackage[]{microtype}
  \UseMicrotypeSet[protrusion]{basicmath} % disable protrusion for tt fonts
}{}
\makeatletter
\@ifundefined{KOMAClassName}{% if non-KOMA class
  \IfFileExists{parskip.sty}{%
    \usepackage{parskip}
  }{% else
    \setlength{\parindent}{0pt}
    \setlength{\parskip}{6pt plus 2pt minus 1pt}}
}{% if KOMA class
  \KOMAoptions{parskip=half}}
\makeatother
\usepackage{xcolor}
\usepackage[margin=1in]{geometry}
\usepackage{color}
\usepackage{fancyvrb}
\newcommand{\VerbBar}{|}
\newcommand{\VERB}{\Verb[commandchars=\\\{\}]}
\DefineVerbatimEnvironment{Highlighting}{Verbatim}{commandchars=\\\{\}}
% Add ',fontsize=\small' for more characters per line
\usepackage{framed}
\definecolor{shadecolor}{RGB}{248,248,248}
\newenvironment{Shaded}{\begin{snugshade}}{\end{snugshade}}
\newcommand{\AlertTok}[1]{\textcolor[rgb]{0.94,0.16,0.16}{#1}}
\newcommand{\AnnotationTok}[1]{\textcolor[rgb]{0.56,0.35,0.01}{\textbf{\textit{#1}}}}
\newcommand{\AttributeTok}[1]{\textcolor[rgb]{0.13,0.29,0.53}{#1}}
\newcommand{\BaseNTok}[1]{\textcolor[rgb]{0.00,0.00,0.81}{#1}}
\newcommand{\BuiltInTok}[1]{#1}
\newcommand{\CharTok}[1]{\textcolor[rgb]{0.31,0.60,0.02}{#1}}
\newcommand{\CommentTok}[1]{\textcolor[rgb]{0.56,0.35,0.01}{\textit{#1}}}
\newcommand{\CommentVarTok}[1]{\textcolor[rgb]{0.56,0.35,0.01}{\textbf{\textit{#1}}}}
\newcommand{\ConstantTok}[1]{\textcolor[rgb]{0.56,0.35,0.01}{#1}}
\newcommand{\ControlFlowTok}[1]{\textcolor[rgb]{0.13,0.29,0.53}{\textbf{#1}}}
\newcommand{\DataTypeTok}[1]{\textcolor[rgb]{0.13,0.29,0.53}{#1}}
\newcommand{\DecValTok}[1]{\textcolor[rgb]{0.00,0.00,0.81}{#1}}
\newcommand{\DocumentationTok}[1]{\textcolor[rgb]{0.56,0.35,0.01}{\textbf{\textit{#1}}}}
\newcommand{\ErrorTok}[1]{\textcolor[rgb]{0.64,0.00,0.00}{\textbf{#1}}}
\newcommand{\ExtensionTok}[1]{#1}
\newcommand{\FloatTok}[1]{\textcolor[rgb]{0.00,0.00,0.81}{#1}}
\newcommand{\FunctionTok}[1]{\textcolor[rgb]{0.13,0.29,0.53}{\textbf{#1}}}
\newcommand{\ImportTok}[1]{#1}
\newcommand{\InformationTok}[1]{\textcolor[rgb]{0.56,0.35,0.01}{\textbf{\textit{#1}}}}
\newcommand{\KeywordTok}[1]{\textcolor[rgb]{0.13,0.29,0.53}{\textbf{#1}}}
\newcommand{\NormalTok}[1]{#1}
\newcommand{\OperatorTok}[1]{\textcolor[rgb]{0.81,0.36,0.00}{\textbf{#1}}}
\newcommand{\OtherTok}[1]{\textcolor[rgb]{0.56,0.35,0.01}{#1}}
\newcommand{\PreprocessorTok}[1]{\textcolor[rgb]{0.56,0.35,0.01}{\textit{#1}}}
\newcommand{\RegionMarkerTok}[1]{#1}
\newcommand{\SpecialCharTok}[1]{\textcolor[rgb]{0.81,0.36,0.00}{\textbf{#1}}}
\newcommand{\SpecialStringTok}[1]{\textcolor[rgb]{0.31,0.60,0.02}{#1}}
\newcommand{\StringTok}[1]{\textcolor[rgb]{0.31,0.60,0.02}{#1}}
\newcommand{\VariableTok}[1]{\textcolor[rgb]{0.00,0.00,0.00}{#1}}
\newcommand{\VerbatimStringTok}[1]{\textcolor[rgb]{0.31,0.60,0.02}{#1}}
\newcommand{\WarningTok}[1]{\textcolor[rgb]{0.56,0.35,0.01}{\textbf{\textit{#1}}}}
\usepackage{longtable,booktabs,array}
\usepackage{calc} % for calculating minipage widths
% Correct order of tables after \paragraph or \subparagraph
\usepackage{etoolbox}
\makeatletter
\patchcmd\longtable{\par}{\if@noskipsec\mbox{}\fi\par}{}{}
\makeatother
% Allow footnotes in longtable head/foot
\IfFileExists{footnotehyper.sty}{\usepackage{footnotehyper}}{\usepackage{footnote}}
\makesavenoteenv{longtable}
\usepackage{graphicx}
\makeatletter
\newsavebox\pandoc@box
\newcommand*\pandocbounded[1]{% scales image to fit in text height/width
  \sbox\pandoc@box{#1}%
  \Gscale@div\@tempa{\textheight}{\dimexpr\ht\pandoc@box+\dp\pandoc@box\relax}%
  \Gscale@div\@tempb{\linewidth}{\wd\pandoc@box}%
  \ifdim\@tempb\p@<\@tempa\p@\let\@tempa\@tempb\fi% select the smaller of both
  \ifdim\@tempa\p@<\p@\scalebox{\@tempa}{\usebox\pandoc@box}%
  \else\usebox{\pandoc@box}%
  \fi%
}
% Set default figure placement to htbp
\def\fps@figure{htbp}
\makeatother
\setlength{\emergencystretch}{3em} % prevent overfull lines
\providecommand{\tightlist}{%
  \setlength{\itemsep}{0pt}\setlength{\parskip}{0pt}}
\setcounter{secnumdepth}{-\maxdimen} % remove section numbering
\usepackage{bookmark}
\IfFileExists{xurl.sty}{\usepackage{xurl}}{} % add URL line breaks if available
\urlstyle{same}
\hypersetup{
  pdftitle={DTP coverage},
  pdfauthor={Tristan Verbeek, 2864049},
  hidelinks,
  pdfcreator={LaTeX via pandoc}}

\title{DTP coverage}
\author{Tristan Verbeek, 2864049}
\date{2025-06-19}

\begin{document}
\maketitle

\section{Set-up your environment}\label{set-up-your-environment}

\begin{Shaded}
\begin{Highlighting}[]
\FunctionTok{library}\NormalTok{(tidyverse)}
\end{Highlighting}
\end{Shaded}

\begin{verbatim}
## -- Attaching core tidyverse packages ------------------------ tidyverse 2.0.0 --
## v dplyr     1.1.4     v readr     2.1.5
## v forcats   1.0.0     v stringr   1.5.1
## v ggplot2   3.5.2     v tibble    3.3.0
## v lubridate 1.9.4     v tidyr     1.3.1
## v purrr     1.0.4     
## -- Conflicts ------------------------------------------ tidyverse_conflicts() --
## x dplyr::filter() masks stats::filter()
## x dplyr::lag()    masks stats::lag()
## i Use the conflicted package (<http://conflicted.r-lib.org/>) to force all conflicts to become errors
\end{verbatim}

\begin{Shaded}
\begin{Highlighting}[]
\FunctionTok{library}\NormalTok{(readxl)}
\FunctionTok{library}\NormalTok{(dplyr)}
\end{Highlighting}
\end{Shaded}

\section{Title Page}\label{title-page}

By: G.J.R. Weydemuller (student number)

L. Waaijenberg (2862154)

P. Gandra Rodrigues (2870539)

Q. Faber (2861921)

R.A. Kali (2850680)

T. Adam (student number)

T.M.V. Verbeek (student number)

Tutor: J.F. Fitzgerald

Tutorial group: Tutorial group 1

Course: Programming for Economists

Institute: Vrije Universiteit Amsterdam

\section{Part 1 - Identify a Social
Problem}\label{part-1---identify-a-social-problem}

Use APA referencing throughout your document.
\href{https://www.mendeley.com/guides/apa-citation-guide/}{Here's a link
to some explanation.}

\subsection{1.1 Describe the Social
Problem}\label{describe-the-social-problem}

\subsection{Why is DTP vaccination a social
problem?}\label{why-is-dtp-vaccination-a-social-problem}

DTP vaccination, protecting against diphtheria, tetanus, and pertussis,
is a cornerstone of global public health. It is one of the most
cost-effective interventions, preventing an estimated 4.4 million deaths
annually (unicef, 2024). Despite its proven effectiveness and inclusion
in nearly all national immunization schedules, millions of children
worldwide remain under-vaccinated or entirely unvaccinated. In 2023
alone, approximately 21 million children were either unvaccinated or
under-vaccinated, with 14.5 million receiving no vaccines at all
(unicef, 2024). This gap in coverage leads to preventable illness,
long-term disability, and death, particularly in low- and middle-income
countries (Our World in Data, sd). The issue is not merely medical but
deeply social, as it reflects and reinforces inequalities in access to
healthcare, education, and infrastructure (unicef, 2024).

Vaccination coverage is also a proxy for the strength of a country's
healthcare system. Low DTP coverage often signals broader systemic
issues such as poverty, conflict, misinformation, and weak governance.
These factors disproportionately affect vulnerable populations, making
DTP vaccination a critical indicator of social justice and equity
(unicef, 2024).

Research question: ``How does government healthcare spending per capita
correlate with national vaccination coverage rates for routine
immunizations in low-, middle-, and high-income countries?''

\subsection{Sources identifying DTP vaccination as a social
problem}\label{sources-identifying-dtp-vaccination-as-a-social-problem}

World Health Organization (WHO) and UNICEF have jointly warned that
vaccine-preventable diseases, including diphtheria, are resurging due to
declining immunization coverage, misinformation, and funding cuts. They
emphasize that millions of children are at risk, and urgent investment
in immunization is needed to prevent a reversal of decades of progress
(WHO Media Team, 2025).

UNICEF outlines the severe consequences of DTP-preventable diseases.
Diphtheria can cause heart and nerve damage, tetanus is often fatal even
with treatment, and pertussis can lead to pneumonia and death in
infants. These diseases disproportionately affect children in
low-resource settings, highlighting the social dimension of
under-vaccination~(UNICEF, sd).

A UNICEF data report on DTP vaccine dropout rates shows that millions of
children start but do not complete the DTP series. This dropout is
especially prevalent in low-income countries and is linked to systemic
barriers such as healthcare access, education, and infrastructure (Wang;
et al., 2019).

\subsection{What has not been fully
researched}\label{what-has-not-been-fully-researched}

While many studies have examined DTP coverage and its determinants,
there is limited comparative analysis across income groups that links
government healthcare spending per capita directly to national DTP
vaccination rates. Most existing research focuses on either micro-level
determinants (e.g., parental education, rural access) or macro-level
trends without integrating economic policy variables. UNICEF notes that
while DTP3 coverage is widely used as a marker of how well countries
provide routine immunization, disparities persist, and economic factors
such as national healthcare investment are often underexplored in
comparative frameworks (UNICEF, 2024).

\subsection{Contribution of this
report}\label{contribution-of-this-report}

This report aims to fill that gap by analyzing how government healthcare
spending correlates with DTP vaccination coverage across low-, middle-,
and high-income countries. This research will provide new insights into
whether increased public investment in health translates into better
immunization outcomes. This could inform both global health policy and
economic development strategies, especially in resource-constrained
settings.

\section{Part 2 - Data Sourcing}\label{part-2---data-sourcing}

\subsection{2.1 Load in the data}\label{load-in-the-data}

Source 1 (DTP vaccination coverage):
\url{https://immunizationdata.who.int/global/wiise-detail-page/diphtheria-tetanus-toxoid-and-pertussis-(dtp)-vaccination-coverage?CODE=AFR+EMR+EUR+AMR+SEAR+WPR&ANTIGEN=&YEAR=}

Source 2 (Government healthcare spending):
\url{https://www.who.int/data/gho/data/indicators/indicator-details/GHO/current-health-expenditure-(che)-per-capita-in-us-dollar}

Source 3 (Child death rates):
\url{https://www.who.int/data/gho/data/indicators/indicator-details/GHO/number-of-under-five-deaths}

\begin{Shaded}
\begin{Highlighting}[]
\FunctionTok{head}\NormalTok{(raw\_healthspending\_data)}
\end{Highlighting}
\end{Shaded}

\begin{verbatim}
##            IndicatorCode                                          Indicator
## 1 GHED_CHE_pc_US_SHA2011 Current health expenditure (CHE) per capita in US$
## 2 GHED_CHE_pc_US_SHA2011 Current health expenditure (CHE) per capita in US$
## 3 GHED_CHE_pc_US_SHA2011 Current health expenditure (CHE) per capita in US$
## 4 GHED_CHE_pc_US_SHA2011 Current health expenditure (CHE) per capita in US$
## 5 GHED_CHE_pc_US_SHA2011 Current health expenditure (CHE) per capita in US$
## 6 GHED_CHE_pc_US_SHA2011 Current health expenditure (CHE) per capita in US$
##   ValueType ParentLocationCode ParentLocation Location.type SpatialDimValueCode
## 1   numeric             GLOBAL         Global    WHO region                 AMR
## 2   numeric             GLOBAL         Global    WHO region                 AFR
## 3   numeric             GLOBAL         Global    WHO region                 WPR
## 4   numeric             GLOBAL         Global    WHO region                SEAR
## 5   numeric             GLOBAL         Global    WHO region                 EUR
## 6   numeric             GLOBAL         Global    WHO region                 EMR
##                Location Period.type Period IsLatestYear Dim1.type Dim1
## 1              Americas        Year   2022         true        NA   NA
## 2                Africa        Year   2022         true        NA   NA
## 3       Western Pacific        Year   2022         true        NA   NA
## 4       South-East Asia        Year   2022         true        NA   NA
## 5                Europe        Year   2022         true        NA   NA
## 6 Eastern Mediterranean        Year   2022         true        NA   NA
##   Dim1ValueCode Dim2.type Dim2 Dim2ValueCode Dim3.type Dim3 Dim3ValueCode
## 1            NA        NA   NA            NA        NA   NA            NA
## 2            NA        NA   NA            NA        NA   NA            NA
## 3            NA        NA   NA            NA        NA   NA            NA
## 4            NA        NA   NA            NA        NA   NA            NA
## 5            NA        NA   NA            NA        NA   NA            NA
## 6            NA        NA   NA            NA        NA   NA            NA
##   DataSourceDimValueCode DataSource FactValueNumericPrefix FactValueNumeric
## 1                     NA         NA                     NA           1256.0
## 2                     NA         NA                     NA            132.6
## 3                     NA         NA                     NA           1364.0
## 4                     NA         NA                     NA            241.0
## 5                     NA         NA                     NA           2938.0
## 6                     NA         NA                     NA            544.2
##   FactValueUoM FactValueNumericLowPrefix FactValueNumericLow
## 1           NA                        NA                  NA
## 2           NA                        NA                  NA
## 3           NA                        NA                  NA
## 4           NA                        NA                  NA
## 5           NA                        NA                  NA
## 6           NA                        NA                  NA
##   FactValueNumericHighPrefix FactValueNumericHigh   Value
## 1                         NA                   NA 1256.40
## 2                         NA                   NA  132.58
## 3                         NA                   NA 1363.92
## 4                         NA                   NA  240.99
## 5                         NA                   NA 2937.83
## 6                         NA                   NA  544.24
##   FactValueTranslationID FactComments Language             DateModified
## 1                     NA           NA       EN 2024-12-09T23:00:00.000Z
## 2                     NA           NA       EN 2024-12-09T23:00:00.000Z
## 3                     NA           NA       EN 2025-04-03T22:00:00.000Z
## 4                     NA           NA       EN 2025-04-03T22:00:00.000Z
## 5                     NA           NA       EN 2024-12-09T23:00:00.000Z
## 6                     NA           NA       EN 2024-12-09T23:00:00.000Z
\end{verbatim}

\begin{Shaded}
\begin{Highlighting}[]
\FunctionTok{head}\NormalTok{(raw\_coverage\_data)}
\end{Highlighting}
\end{Shaded}

\begin{verbatim}
## # A tibble: 6 x 11
##   GROUP       CODE  NAME      YEAR ANTIGEN ANTIGEN_DESCRIPTION COVERAGE_CATEGORY
##   <chr>       <chr> <chr>    <dbl> <chr>   <chr>               <chr>            
## 1 WHO_REGIONS AFR   African~  2023 DTPCV3  DTP-containing vac~ WUENIC           
## 2 WHO_REGIONS EMR   Eastern~  2023 DTPCV3  DTP-containing vac~ WUENIC           
## 3 WHO_REGIONS EUR   Europea~  2023 DTPCV3  DTP-containing vac~ WUENIC           
## 4 WHO_REGIONS AMR   Region ~  2023 DTPCV3  DTP-containing vac~ WUENIC           
## 5 WHO_REGIONS SEAR  South-E~  2023 DTPCV3  DTP-containing vac~ WUENIC           
## 6 WHO_REGIONS AFR   African~  2022 DTPCV3  DTP-containing vac~ WUENIC           
## # i 4 more variables: COVERAGE_CATEGORY_DESCRIPTION <chr>, TARGET_NUMBER <dbl>,
## #   DOSES <dbl>, COVERAGE <dbl>
\end{verbatim}

\begin{Shaded}
\begin{Highlighting}[]
\NormalTok{inline\_code }\OtherTok{=} \ConstantTok{TRUE}
\end{Highlighting}
\end{Shaded}

\subsection{2.2 Provide a short summary of the
dataset(s)}\label{provide-a-short-summary-of-the-datasets}

\textbf{Source 1: DTP Vaccination Coverage}

- Credibility: This dataset is published by the WHO/UNICEF Estimates of
National Immunization Coverage (WUENIC), the leading global authority on
public health. WHO collaborates with national governments and UNICEF to
compile and validate immunization data (World Health Organization,
2024).

- Data Quality: The data is based on a combination of administrative
data, household surveys, and expert review. This triangulation improves
accuracy and reliability.

- Relevance: It directly provides the DTP3 coverage rates, which are
central to your research question. The data is also disaggregated by
region and year, allowing for temporal and cross-regional analysis.

\textbf{Source 2: Government Healthcare Spending}

- Credibility: The World Bank is a globally respected institution that
provides standardized economic and development data. It sources health
expenditure data from the WHO Global Health Expenditure Database (World
Bank, 2024).

- Data Quality: The data is collected using internationally accepted
methodologies and is updated annually. It includes consistent
country-level indicators, making it ideal for cross-country comparisons.

- Relevance: This dataset provides the independent variable in your
analysis---government health spending per capita---which you are
correlating with vaccination coverage.

\textbf{Source 3: Child Death Rates}

- Credibility: This data is also from the World Health Organization,
specifically its Global Health Observatory (GHO), which is the main
repository for WHO's health-related statistics (World Health
Organization, 2024).

- Data Quality: The under-five mortality data is compiled from national
vital registration systems, surveys, and statistical modeling, and is
reviewed by global health experts.

- Relevance: While not part of your main correlation analysis, this
dataset provides valuable contextual insight into the broader social
impact of vaccination coverage and healthcare investment.

\subsection{Complementarity of the
Datasets}\label{complementarity-of-the-datasets}

These datasets are complementary in structure and purpose:

· The~DTP dataset~provides the~dependent variable~(vaccination
coverage).

· The~health spending dataset~provides the~independent
variable~(government health expenditure per capita).

· The child death dataset provides the independent variable (child
death)

Together, they allow for a cross-country analysis of how public
investment in health and child death correlates with immunization
outcomes.

\subsection{Suitability for the Research
Topic}\label{suitability-for-the-research-topic}

The research question investigates the relationship between government
healthcare spending and DTP vaccination coverage across income groups.
These datasets are ideal because:

· They are~comprehensive and global, enabling cross-country comparisons.

· They are~updated annually, supporting time-series or cross-sectional
analysis.

· They are~standardized and well-documented, ensuring data quality and
reproducibility.

\subsection{Data Limitations}\label{data-limitations}

Despite their strengths, the datasets have limitations:

WHO DTP Coverage \& WHO child death:

· Some countries have~missing or estimated values, especially in
conflict zones.

· Coverage estimates may be influenced by~reporting biases~or~survey
recall errors.

· The data may not fully reflect~subnational disparities~in vaccination
access.

World Bank Health Spending:

· Spending data are in~current US dollars, which may be affected by
inflation and exchange rate fluctuations.

· The data do not indicate~how funds are allocated~(e.g., toward
immunization vs.~other services).

· Some countries have~incomplete or outdated records~for recent years.

These limitations mean that while the analysis can reveal correlations,
it cannot establish causality or account for all contextual factors.

\subsection{2.3 Describe the type of variables
included}\label{describe-the-type-of-variables-included}

WHO DTP Vaccination Coverage:

\begin{itemize}
\item
  Antigen: DTP 1 and DTP3
\item
  Coverage: vaccinated target population
\item
  Units: percentage
\item
  Target\_Pop: target population for the given vaccine
\item
  Years Covered: 1980--2023
\item
  Region: WHO regions
\item
  Frequency: Annual
\item
  Source: WHO/UNICEF Joint Reporting Form and surveys (e.g., DHS, MICS)
\item
  Estimation Method: Combination of administrative data, surveys, and
  expert review.
\end{itemize}

World Bank Health Spending:

\begin{itemize}
\item
  Indicator: Current health expenditure per capita
\item
  Years Covered: 2000--2023
\item
  Region: WHO regions
\item
  Frequency: Annual
\item
  Units: US dollars
\item
  Source: WHO Global Health Expenditure
  Database\hyperref[_msocom_1]{{[}KR(1{]}}~
\end{itemize}

WHO child deaths among children under five:

\begin{itemize}
\item
  Indicator: Mortality rate under children under five
\item
  Target\_Pop: population of children aged 0--4 years
\item
  Years covered: 1980-2023
\item
  Region: WHO regions
\item
  Frequency: Annual
\item
  Source: WHO/UNICEF GHO, national civil registration, surveys (DHS,
  MICS)
\item
  Estimation method: Combination of available national data (CRVS,
  surveys) and adjustments by statistical methods and expert review.
\end{itemize}

\section{Part 3 - Quantifying}\label{part-3---quantifying}

\subsection{3.1 Data cleaning}\label{data-cleaning}

\subsubsection{Health Spending Data}\label{health-spending-data}

We began by selecting the relevant columns from the raw health spending
dataset and renaming them for clarity. We also ensured that the
\texttt{Year} and \texttt{Spending} columns were numeric to support
time-based analysis. This step was essential to align the data with the
DTP vaccination dataset, which also uses \texttt{Region} and
\texttt{Year} as key identifiers.

\subsubsection{DTP Vaccination Coverage
Data}\label{dtp-vaccination-coverage-data}

Next, we filtered the immunization dataset to include only WHO regions
and WUENIC estimates, which are considered the most reliable. We also
created a new variable, \texttt{PeriodGroup}, to group years into
meaningful intervals for trend analysis. This transformation allows us
to analyze changes in vaccination coverage over time in a more
interpretable way.

\subsubsection{Child Mortality Data}\label{child-mortality-data}

We also prepared the child mortality dataset by selecting and renaming
relevant columns, and converting values to numeric types. This ensures
that the dataset can be merged with the others on \texttt{Region} and
\texttt{Year}.

\subsubsection{Error Handling and Fixes}\label{error-handling-and-fixes}

During the cleaning process, we encountered several common issues:

\begin{itemize}
\tightlist
\item
  \textbf{Missing values (\texttt{NA})}: These were removed using
  \texttt{na.omit()} or filtered out using \texttt{filter(!is.na(...))}
  to ensure clean merges.
\item
  \textbf{Non-numeric values}: Some columns were read as character types
  and had to be converted using \texttt{as.numeric()}.
\item
  \textbf{Inconsistent region names}: We verified that region names
  matched across datasets. In an ideal scenario, standardized region
  codes (e.g., ISO or WHO codes) would be used to avoid ambiguity.
\item
  \textbf{Data alignment}: We ensured that all datasets used the same
  time format and level of aggregation (annual, by WHO region).
\end{itemize}

\subsection{3.2 Generate necessary
variables}\label{generate-necessary-variables}

To analyze trends over time, we created a \textbf{period variable} for
each dataset to group years into meaningful intervals. This allows us to
compare average values across consistent time blocks.

\subsubsection{Health Spending}\label{health-spending}

We created a new variable \texttt{Period} in the health spending dataset
using \texttt{mutate()} and \texttt{case\_when()}. We then calculated
the \textbf{average health spending} per region and period.

\begin{itemize}
\tightlist
\item
  \textbf{Variables used}: \texttt{Spending}, \texttt{Region},
  \texttt{Period}~
\item
  \textbf{Purpose}: Calculates average health spending per region and
  period.~
\item
  \textbf{Usefulness}: Highlights regional differences and trends in
  health investment.~
\item
  \textbf{Intended analysis}: Visualize and compare health spending
  across time and regions.
\end{itemize}

\subsubsection{DTP Vaccination Coverage}\label{dtp-vaccination-coverage}

For the DTP dataset, we used an existing variable \texttt{PeriodGroup}
(created earlier) and calculated the \textbf{average vaccination
coverage} per region and period.

\begin{itemize}
\tightlist
\item
  \textbf{Variables used}: \texttt{Coverage}, \texttt{Region},
  \texttt{PeriodGroup}~
\item
  \textbf{Purpose}: Computes average DTP vaccine coverage per region and
  period.~
\item
  \textbf{Usefulness}: Evaluates immunization program effectiveness.~
\item
  \textbf{Intended analysis}: Track vaccine coverage trends and relate
  to health outcomes
\end{itemize}

\subsubsection{Child Mortality}\label{child-mortality}

We also created a \texttt{Period} variable for the child mortality
dataset. We then summarized the \textbf{average number of child deaths}
per region and period:

\begin{itemize}
\tightlist
\item
  \textbf{Variables used}: \texttt{Deaths}, \texttt{Region},
  \texttt{Period}\\
\item
  \textbf{Purpose}: Calculates average child deaths per region and
  period.\\
\item
  \textbf{Usefulness}: Measures health outcomes and can be linked to
  spending and coverage.\\
\item
  \textbf{Intended analysis}: Identify regions with high or declining
  child mortality and explore correlations.
\end{itemize}

These variables allow us to compare trends in health spending,
vaccination coverage, and child mortality across WHO regions and over
time.

\section{3.25 merge datasets together}\label{merge-datasets-together}

\subsection{3.3 Visualize temporal
variation}\label{visualize-temporal-variation}

\begin{verbatim}
## Warning: Using `size` aesthetic for lines was deprecated in ggplot2 3.4.0.
## i Please use `linewidth` instead.
## This warning is displayed once every 8 hours.
## Call `lifecycle::last_lifecycle_warnings()` to see where this warning was
## generated.
\end{verbatim}

\pandocbounded{\includegraphics[keepaspectratio]{Template_Assignment_files/figure-latex/unnamed-chunk-11-1.pdf}}
\pandocbounded{\includegraphics[keepaspectratio]{Template_Assignment_files/figure-latex/unnamed-chunk-11-2.pdf}}

\subsubsection{Temporal Trends in Health Investment and
Outcomes}\label{temporal-trends-in-health-investment-and-outcomes}

These two line graphs visualize how healthcare efficiency and outcomes
have evolved across WHO regions over five-year periods from 2000 to
2022. They align closely with the research question by showing how
healthcare spending relates to vaccination coverage and child mortality
over time.

\paragraph{Graph 1: Child Deaths per Spending over
Time}\label{graph-1-child-deaths-per-spending-over-time}

\begin{itemize}
\tightlist
\item
  \textbf{What it shows}: The number of child deaths per unit of
  healthcare spending.
\item
  \textbf{Trend}: A clear downward trend across all regions, indicating
  improved outcomes per dollar spent.
\item
  \textbf{Notable pattern}: Africa and Europe show the most significant
  declines, suggesting major gains in healthcare efficiency or
  effectiveness.
\item
  \textbf{Interpretation}: This trend reflects global progress in
  reducing child mortality, likely due to better-targeted spending,
  improved healthcare delivery, and expanded immunization coverage.
\end{itemize}

\paragraph{Graph 2: Dollar per Coverage Point over
Time}\label{graph-2-dollar-per-coverage-point-over-time}

\begin{itemize}
\tightlist
\item
  \textbf{What it shows}: The average cost (in USD) required to increase
  DTP3 vaccination coverage by one percentage point.
\item
  \textbf{Trend}: Most regions show a relatively stable or slightly
  increasing cost over time.
\item
  \textbf{Notable pattern}: The Western Pacific region shows a
  significant increase in cost per coverage point in recent years,
  suggesting rising marginal costs or reduced efficiency.
\item
  \textbf{Interpretation}: While some regions maintain
  cost-effectiveness, others may be facing diminishing returns on
  investment in immunization programs.
\end{itemize}

\paragraph{Comparison and Dynamics}\label{comparison-and-dynamics}

\begin{itemize}
\tightlist
\item
  \textbf{Efficiency vs.~Outcome}: While the cost of achieving
  additional vaccination coverage is rising or stabilizing, the impact
  of spending on reducing child deaths is improving.
\item
  \textbf{Temporal Insight}: These trends highlight the evolving
  dynamics of global health investment---where spending may be
  increasing, but so is its effectiveness in saving lives.
\end{itemize}

\paragraph{Implications}\label{implications}

\begin{itemize}
\tightlist
\item
  Policymakers should consider both cost and outcome metrics when
  evaluating healthcare strategies.
\item
  The rising cost per coverage point in some regions may call for more
  efficient delivery models or targeted interventions.
\item
  The declining child deaths per spending metric is a positive signal of
  global health progress and a justification for continued investment.
\end{itemize}

\subsection{3.4 Visualize spatial
variation}\label{visualize-spatial-variation}

\pandocbounded{\includegraphics[keepaspectratio]{Template_Assignment_files/figure-latex/unnamed-chunk-13-1.pdf}}

\subsubsection{Spatial Analysis of DTP3 Vaccination Coverage
(2020--2022)}\label{spatial-analysis-of-dtp3-vaccination-coverage-20202022}

This world map visualizes average DTP3 vaccination coverage by WHO
region from 2020 to 2022, using a color gradient from purple (60\%) to
yellow (100\%). The map reveals clear regional disparities: high-income
regions such as Europe and the Western Pacific generally exhibit higher
coverage (light green to yellow), while lower-income regions like Africa
and parts of the Eastern Mediterranean show lower coverage (purple to
blue).

\paragraph{Spatial Anomalies and Regional
Comparisons}\label{spatial-anomalies-and-regional-comparisons}

Notably, some middle-income regions, such as parts of South-East Asia,
achieve relatively high coverage despite lower spending levels,
suggesting efficient immunization programs. Conversely, certain
countries in the Americas show unexpectedly moderate coverage despite
higher average spending, indicating potential inefficiencies or access
barriers.

\paragraph{Interpretation and Mapping
Critique}\label{interpretation-and-mapping-critique}

The map effectively communicates broad regional trends but may obscure
within-region variation due to aggregation. For example, large countries
with internal disparities (e.g., India, Brazil) are represented by a
single color, which may oversimplify complex national dynamics.
Additionally, the use of discrete color bands may exaggerate differences
between countries near category thresholds.

\paragraph{Mapping Choices}\label{mapping-choices}

While the color scheme is intuitive and accessible, a continuous
gradient or country-level granularity could enhance interpretability.
Including data labels or interactive elements (in a web-based version)
would further support detailed analysis. Despite these limitations, the
map serves as a compelling visual summary of global immunization equity
and supports the research question by highlighting spatial patterns in
vaccination coverage relative to regional healthcare investment.

\subsection{3.5 Visualize sub-population
variation}\label{visualize-sub-population-variation}

\pandocbounded{\includegraphics[keepaspectratio]{Template_Assignment_files/figure-latex/visualise_map-1.pdf}}

\subsubsection{Subgroup Analysis: Top vs Bottom 3 WHO Regions
(2022)}\label{subgroup-analysis-top-vs-bottom-3-who-regions-2022}

This bar chart compares the top 3 and bottom 3 WHO regions based on
average healthcare spending per capita in 2022. It visualizes three key
metrics:

\begin{itemize}
\tightlist
\item
  \textbf{Average Health Spending (USD per capita)}
\item
  \textbf{Average DTP3 Vaccination Coverage (\%)}
\item
  \textbf{Average Child Deaths}
\end{itemize}

\paragraph{Alignment with Research
Topic}\label{alignment-with-research-topic}

This visualization directly supports the research question:\\
\emph{``How does government healthcare spending per capita correlate
with national vaccination coverage rates for routine immunisations?''}

By dividing regions into high- and low-spending groups, the plot
highlights how investment levels relate to health outcomes across
sub-populations.

\paragraph{Group Differences
(Quantified)}\label{group-differences-quantified}

\begin{longtable}[]{@{}llll@{}}
\toprule\noalign{}
Metric & Top 3 Regions & Bottom 3 Regions & Difference \\
\midrule\noalign{}
\endhead
\bottomrule\noalign{}
\endlastfoot
Spending & \$1852.7 & \$305.9 & +\$1546.8 \\
Coverage & 91\% & 82\% & +9\% \\
Child Deaths & 149,154.7 & 1,516,736.3 & −1,367,581.6 \\
\end{longtable}

\begin{itemize}
\tightlist
\item
  Top 3 regions spend over \textbf{6 times more} per capita on
  healthcare.
\item
  They achieve \textbf{9 percentage points higher} vaccination coverage.
\item
  They experience nearly \textbf{10 times fewer} child deaths.
\end{itemize}

\paragraph{Evaluation of Subgroup
Analysis}\label{evaluation-of-subgroup-analysis}

This subgroup comparison reveals a clear gradient: higher spending is
associated with better vaccination outcomes and significantly lower
child mortality. The differences are not only statistically meaningful
but also socially and ethically significant, especially in the context
of global health equity.

\paragraph{Implications}\label{implications-1}

\begin{itemize}
\tightlist
\item
  \textbf{Policy Insight}: The data suggests that increasing healthcare
  investment in underfunded regions could substantially improve
  immunization coverage and reduce child mortality.
\item
  \textbf{Equity Concern}: The stark contrast in child death rates
  underscores the urgent need for targeted support in low-spending
  regions.
\item
  \textbf{Further Research}: This analysis invites deeper exploration
  into how efficiently funds are used and what other systemic factors
  (e.g., governance, infrastructure) influence outcomes.
\end{itemize}

\subsection{3.6 Event analysis}\label{event-analysis}

\pandocbounded{\includegraphics[keepaspectratio]{Template_Assignment_files/figure-latex/unnamed-chunk-15-1.pdf}}
\pandocbounded{\includegraphics[keepaspectratio]{Template_Assignment_files/figure-latex/unnamed-chunk-15-2.pdf}}
Applying the findings related to the correlation between government
healthcare spending per capita and national vaccination coverage rates
worldwide, we can analyse a critical event period from 2020 to 2022
which relates to the COVID-19 pandemic. Unlike gradual policy reforms
during the 2010-2014 period such as GAVI's (source
\url{https://www.gavi.org/our-alliance/strategy/phase-3-2011-2015})
innovative financing mechanisms which introduced market-based incentives
to lower vaccine pricing and availability structure, the pandemic
created an acute simultaneous shock to every healtchare system
worldwide. This situation enables a unique opportunity to examine how
external crises affect government healthcare spending per capita and
national vaccination coverage rates worldwide.

During COVID-19 routine immunizations were suspended or decreased as
worldwide healthcare systems had to respond to the pandemic, meaning
that regardless of funding levels directed to vaccinations, their
coverage still dropped. (source
\url{https://www.who.int/news/item/22-05-2020-at-least-80-million-children-under-one-at-risk-of-diseases-such-as-diphtheria-measles-and-polio-as-covid-19-disrupts-routine-vaccination-efforts-warn-gavi-who-and-unicef}).

Due to global lockdowns and border restrictions another major disruption
was on the supply chain, as the struggle to transport vaccinations
across borders and maintain cold storage systems increased, leading to
decreased vaccination availability. Increasingly relevant to this study,
a decrease in 7.7\% DTP3 vaccine coverage worldwide was recognized
compared to expected doses delivered in the absence of COVID. (source
\url{https://www.thelancet.com/article/S0140-6736\%2821\%2901337-4/fulltext})

The child deaths per spending data reveals a steady improvement
trajectory up until before COVID-19 where regions started to show signs
of stagnation. African regions kept stable child deaths per spending
through the pandemic which suggests this region had resilience
strategies such as communtiy health worker programs which focused on
under-five care while more formal systems were focused on COVID related
health issues. Regardless of staffing shortages and movement
restriction, these pre-established networks prevented many childrens'
deaths in this region. (source
\url{https://pubmed.ncbi.nlm.nih.gov/38963883/}) This data might however
have overstated results as it may reflect COVID-19 related deaths and
healthcare access disruptions which caused an increased number of infant
mortality, complciating the interepretation of spending efficiency
across this time period.

The global coverage map strucks with regional disparities during the
2020-2022 period, ranging from 60\% to close to 100\% in some WHO
regions. This suggests that pre-existing health care systems' resilience
and adaptive mechanisms worldwide varied greatly upon facing the
pandemic. Whereas most childhood vaccines declined in coverage
worldwide, it was more prominent in low and lower-middle income
countries (source
\url{https://pmc.ncbi.nlm.nih.gov/articles/PMC10249397/}?) , which
reinstates that more developed countries with better precautionary
health systems can keep a more stable vaccination coverage even during
more strenuous moments on the healthcare system.

Similarly the comparison between subpopulations shows how bottom
performing regions, namely africa, maintained a 82\% coverage during the
pandemic while top performing regions achieved 91\%. This data aligns
with the previous graph, showing that stronger health systems in Europe
can much better mitigate coverage maintenance challenges compared to
African regions.

Finally, the dollar per coverage point data shows a sudden spike in
costs, which suggests that it became much more expensive to maintain
coverage during the pandemic. The aforementioned supply chain disruption
provides an explanation to this spike. Service suspensions and
supply-chain delays caused vaccination volumes to drop, meaning that the
delivery costs such as personnel and cold chains had to be spread over
fewer doses, signficantly increasing the per-point cost.

Here you provide a description of why the plot above is relevant to your
specific social problem.

\section{Part 4 - Discussion}\label{part-4---discussion}

Over the course of the analysis of the relation between all four graphs
and COVID-19, strong temporal evidence has come to light, hinting a
pandemic-related causation rather than coincidental trends. This
challenges theories which assume stable and linear relationships between
government healthcare spending and vaccination coverages. Three key
insights emerge from the analysis:

Crisis-state exception: Traditional vaccination coverage models operate
under the assumption of stability and predictability, meaning that major
disruptions such as COVID-19 can crumble healthcare systems if these are
not equipped with dynamic frameworks addressing both normal and
crisis-states.

System resilience hierarchy: An unrecognized hiearchy of system
resilience between regions emerged with this study, slicing through
traditional income classifications. Community-based delivery systems
such as the ones in Rwanda and Bangladesh (source
\url{https://pubmed.ncbi.nlm.nih.gov/38963883/}) demonstrated certain
poorer areas have superior crisis adaptability compared to other highly
bureacratic systems in richer countries. This insights allows for the
discussion diving into the potential vulnerability that comes with
highly institutionalized systems.

Resource Mobilization Capacity: The cost spike related to supply chain
disruptions highlights the importance of rapidly scaling spending in
response to crises. Similar to Keynes' view, this insight challenges
static spending-outcome models by introducing varied increase in
resources as a better option.

One key takeaway from the study is how resilient delivery systems are
key to react to sudden disruptors. The superior performance of
community-based systems might lead to conversations on how even more
developed countries should emphasize on decentralized,
locally-controlled delivery mechanisms to maintain healthcare standards
during major events.

Furthemore over-sophisticated health systems and its shortcomings could
direct policy makers to evaluate the efficiency advantages or
disadvantages of complex and integrated systems. Finally, the varied
regional responses to the pandemic era disclose poor international
coordination, with each region managing responses independently.
Advocating once more for more flexible, decentralized international
response frameworks.

\subsection{4.1 Discuss your findings}\label{discuss-your-findings}

Our most significant finding was that sophisticated, well-funded
healthcare systems suffered the greatest efficiency losses during the
pandemic, whereas simpler, less-funded systems proved resilience. Europe
maintained \textgreater90\% coverage but had dramatic cost increases
from \$25 to \$31 per coverage point during COVID. On the other hand,
african regions maintained stable child deaths per spending ratios
despite having the lowest spending levels.

Traditional income classifications were deemed flawed to predict
pandemic resilience. In turn, system design characteristics such as
degree of centralization and institutional complexity became the main
differentiator between regions' adaptability to extremely impactful
events. Community-based systems' adaptive capacity prevailed over highly
institutionalized systems' tendency to create multiple failure points,
requiring major resource mobilization.

Pre-pandemic data showed a predictable improvement amongst all regions,
but when the pandemic struck, dramatic changes were observed that
altered spending-outcome relationships.

The findings of this study challenge core health economics assumptions
and suggests the need for dynamic models, capable to adapt to stable and
unstable environments. For policy implementation, the superior
performance of community-based systems drags the topic of pandemic
preparedness to a more decentralized approach.

\section{Part 5 - Reproducibility}\label{part-5---reproducibility}

\subsection{5.1 Github repository link}\label{github-repository-link}

Provide the link to your PUBLIC repository here: \ldots{}

\subsection{5.2 Reference list}\label{reference-list}

\textbf{References}

Our World in Data. (n.d.). \emph{Deaths caused by vaccine-preventable
diseases, World}. Retrieved from Our World in Data:
\url{https://ourworldindata.org/grapher/deaths-caused-by-vaccine-preventable-diseases-over-time}

unicef. (2024, July). \emph{Immunization}. Retrieved from unicef:
\url{https://data.unicef.org/topic/child-health/immunization/}

UNICEF. (2024, March). \emph{Vaccination \& Immunization Statistics -
UNICEF Data}. Retrieved from UNICEF:
\url{https://knowledge.unicef.org/resource/vaccination-immunization-statistics-unicef-data}

UNICEF. (n.d.). \emph{Vaccines and the diseases they prevent}. Retrieved
from UNICEF:
\url{https://www.unicef.org/parenting/health/vaccines-and-diseases-they-prevent}

WHO. (2023). \emph{global.} Retrieved from immunizationdata:
\url{https://immunizationdata.who.int/global/wiise-detail-page/diphtheria-tetanus-toxoid-and-pertussis-(dtp)-vaccination-coverage?CODE=AFR+EMR+EUR+AMR+SEAR+WPR&ANTIGEN=&YEAR=}

WHO. (2023). \emph{Indicators.} Retrieved from who:
\url{https://www.who.int/data/gho/data/indicators/indicator-details/GHO/number-of-under-five-deaths}

WHO. (2025). \emph{Health expenditure.} Retrieved from who:
\url{https://www.who.int/data/gho/data/indicators/indicator-details/GHO/current-health-expenditure-(che)-per-capita-in-us-dollar}

WHO Media Team. (2025, April 24). \emph{Increases in vaccine-preventable
disease outbreaks threaten years of progress, warn WHO, UNICEF, Gavi} .
Retrieved from World Health Organization:
\url{https://www.who.int/news/item/24-04-2025-increases-in-vaccine-preventable-disease-outbreaks-threaten-years-of-progress--warn-who--unicef--gavi}

Xinhu Wang; Mamadou Diallo. (2019, September 24). \emph{UNICEF}.
Retrieved from How tracking DTP vaccine dropout rates helps protect
children from preventable diseases:
\url{https://data.unicef.org/data-for-action/tracking-dtp-vaccine-dropout-rates-protects-children-from-preventable-diseases/}

\end{document}
